\section{Introduction}

This analysis note presents the measurements of two-particle angular correlations of charged hadrons produced in $e^+e^-$ collisions as a function of charged hadron multiplicity, using 730 $pb^{-1}$ of archived data collected between 91 and 209 GeV with the ALEPH detector at LEP.  Two-particle correlations in high-energy collisions provide valuable information for characterizing Quantum Chromodynamics and have been studied previously for a broad range of collision energies in proton-proton (pp)~\cite{Khachatryan:2010gv}, proton-nucleus (pA)~\cite{CMS:2012qk,Abelev:2012ola,Aad:2012gla}, and nucleus-nucleus (AA)~\cite{Aamodt:2010pa,Chatrchyan:2012wg} collisions. Such measurements can elucidate the underlying mechanism of particle production and reveal possible collective effects resulting from the high particle densities accessible in these collisions.

Studies of two-particle angular correlations in pp, pA and AA collisions are typically performed using two-dimensional $\Delta\eta-\Delta\phi$ correlation functions, where $\Delta\phi$ is the difference in the azimuthal angle $\phi$ between the two particles and $\Delta\eta$ is the difference in pseudorapidity $\eta = -\ln(\tan(\theta/2))$. The polar angle $\theta$ is defined relative to the counterclockwise hadron beam direction.

Of particular interest in studies of collective effects is the long-range (large $|\Delta\eta|$) structure of the two-particle correlation functions. In this region, the function is less susceptible to other known sources of correlations such as resonance decays and fragmentation function of energetic jets. Measurements in high-energy AA collisions have shown significant modification of the long-range structure compared with minimum-bias pp collisions, over a very wide range of collision energies~\cite{Back:2004je,Arsene:2004fa,Adcox:2004mh,Adams:2005dq}. The long-range correlations are commonly interpreted as a consequence of the hydrodynamical flow of the produced strongly interacting medium~\cite{Ollitrault:1992bk} and usually characterized by the Fourier components of the azimuthal particle distributions. The extraction of the second and third Fourier components, usually referred to as elliptic and triangular flow, is of great interest because it is closely related to initial collision geometry and its fluctuation~\cite{Alver:2010gr}. Those measurements allow the extraction of the fundamental transport properties of the medium using hydrodynamic models.

Recently, measurements in pp~\cite{Khachatryan:2010gv} and pPb collisions~\cite{CMS:2012qk,Abelev:2012ola,Aad:2012gla} have revealed the emergence of long-range, near-side ($\Delta\phi\sim 0$) correlations in the selection of collisions with very high number of final state particles. This ``ridge-like'' correlation has inspired a large variety of theoretical models~\cite{Bzdak:2013zma,Dusling:2015gta}. The physical origin of the phenomenon is not yet fully understood. Moreover, it was found that the elliptic flow signal exists even at the lowest nucleon-nucleon center-of-mass energy of 7.7 GeV in AA collisions at the Relativistic Heavy Ion Collider~\cite{Adamczyk:2012ku}. 

Due to the complexity of the hadron-hadron collisions, possible initial state correlations of the partons, such as those arise from color-glass condensate~\cite{Gelis:2010nm, Dusling:2013qoz}, could complicate the interpretation of the pp and pA data. Studies of high multiplicity $e^+e^-$ collision, where the initial kinematics of the collisions are well-controlled, could bring significant insights about the observed phenomenon. These measurements will also enable a direct comparison between different collision systems for the first time. The studies of ridge signal in $e^+e^-$ collisions will bring significant impact to the field of relativistic heavy ion collisions, either change completely the interpretation of the ridge in pp, pA and AA collisions if a significant signal is observed, or serve as an important reference for the final state effect observed in high multiplicity hadron-hadron scatterings if no long-range correlation signal was detected. 

With the archived data from ALEPH collaboration, the correlation functions are studied over a broad range of pseudorapidity $\eta$ (rapidity $y$) and azimuthal angle $\phi$ with respect to the electron-positron beam axis. In addition, the correlation functions are also studied using azimuthal and polar angles calculated with respect to the the event thrust axis and a dijet axis in order to follow the direction of the extending color strings. 
