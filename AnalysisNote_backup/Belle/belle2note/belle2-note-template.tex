\documentclass[preprint,prd,tightenlines]{revtex4}

\usepackage{graphicx} % Include figure files
\usepackage{dcolumn}  % Align table columns on decimal point
\usepackage{colordvi}
\usepackage{color}
\usepackage{epstopdf}
\usepackage{amssymb}
\usepackage{url}
\graphicspath{{ps}}
\usepackage{hyperref}
\usepackage{tabularx}

\renewcommand{\arraystretch}{1.1}

\renewcommand{\thesection}{\arabic{section}}
\renewcommand{\thesubsection}{\thesection.\arabic{subsection}}
\renewcommand{\thesubsubsection}{\thesubsection.\arabic{subsubsection}}

\input ../belle2sym.tex

\begin{document}

%place for definitions and newcommands
\def\belletwo {\it {Belle}}



\vspace*{-3\baselineskip}
\resizebox{!}{3cm}{\includegraphics{belle-logo.eps}}

\vspace*{-5\baselineskip}
\begin{flushright}
BELLE2-NOTE-XX-YYYY-ZZZ\\
DRAFT Version 1.0 \\
\today
\end{flushright}


\title { \quad\\[0.5cm] Title of Belle Note}


\author{X.~ZYW}
\email{XZYW@belle2.kek.jp}
\affiliation{Institute, city, country}% add more lines in case of multiple authors

\author{A.~EIO}
\email{AEIO@belle2.kek.jp}
\affiliation{Institute, city, country}% add more lines in case of multiple authors


\collaboration{The Belle Collaboration}
\noaffiliation

\begin{abstract}
Abstract of B2N... 

\keywords{Belle, ...}
\end{abstract}

\pacs{}

\maketitle

{\renewcommand{\thefootnote}{\fnsymbol{footnote}}}
\setcounter{footnote}{0}

\tableofcontents

\section{Introduction}

This is a template for {\belletwo} internal notes. See some notes on usage of 
references in~\ref{sec:ref}, compilation (\ref{sec:com}) and figures (\ref{sec:fig}). 

\subsection{References}
\label{sec:ref}

Use BibTex for references. Usage:\\
at the end of the main {\tt .tex} file, before {\tt $\backslash$end\{document\}}, 
put:\\
{\tt 
$\backslash$bibliography\{bibfile\}\\
$\backslash$bibliographystyle\{unsrt\}},\\
where {\tt bibfile.bib} is the file in which the references are listed (see example {\tt bib2.bib}), 
and {\tt unsrt} is the style providing for the references to appear in the numbered order 
(i.e. starting from [1]; this is not true, though, for the footnotes, which appear at the 
end of references). The format of each reference in the {\tt bibfile.bib} is directly provided 
by inSPIRE, clicking on the BibTeX link. 

In the bibliography example file the {\belletwo} TDR \cite{Abe:2010sj} reference is included, as well as 
the Geant 4 \cite{Agostinelli:2002hh} and the physics case paper \cite{Aushev:2010bq} references. A bibliography style file has been included in this package for consistent formatting of references, and to include full length titles.

\subsection{Compiling}
\label{sec:com}

In order to produce a {\tt .pdf} file one needs to:\\
{\tt 
- pdflatex filename.tex\\
- bibtex filename.aux (repeat 2x or 3x)\\
- pdflatex filename.tex\\
}

\subsection{Figures}
\label{sec:fig}

Figures are included as usually, shown in the example of Fig.~\ref{fig1} below. Beside the pdf also jpg and eps 
formats can be used. Specifiy the figure files in \\
{\tt 
$\backslash$includegraphics[width=8cm]\{belle-logo.eps\}\\
}
without extensions ({\tt pdflatex} command will take care of that, by using {\tt epstopdf} package 
for the eps files, for example). 

\begin{figure}
\begin{center}
\includegraphics[width=8cm]{belle-logo.eps}
  \caption{Some figure.}
  \label{fig1}
\end{center}
\end{figure}

If you are creating figures from Root, your are advised to use the Belle2 style files under \url{documents/belle2style}. A README file contained in the repository provides instructions for their use.

\subsection{Standard symbols for use in Belle notes and papers}
\label{sec:sym}
A file containing standard symbols is provided in the package \url{belle2sym.tex}, used as follows: \epem \ra \epemg. This is the recommended symbol file for use in Belle notes.

\bibliography{belle2}
\bibliographystyle{belle2-note}

\end{document}

%%% Local Variables:
%%% mode: latex
%%% TeX-master: t
%%% End:
