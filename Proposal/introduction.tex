\section{Introduction}

This paper proposes measurements of two-particle angular correlations of charged hadrons produced in $e^+e^-$ collisions, as a function of charged hadron multiplicity. Two-particle correlations in high-energy collisions provide valuable information for characterizing Quantum Chromodynamics and have been studied previously for a broad range of collision energies in proton-proton (pp)~\cite{Khachatryan:2010gv}, proton-nucleus (pA)~\cite{CMS:2012qk,Abelev:2012ola,Aad:2012gla}, and nucleus-nucleus (AA)~\cite{Aamodt:2010pa,Chatrchyan:2012wg} collisions. Such measurements can elucidate the underlying mechanism of particle production and reveal possible collective effects resulting from the high particle densities accessible in these collisions.

Studies of two-particle angular correlations are typically performed using two-dimensional $\Delta\eta-\Delta\phi$ correlation functions, where $\Delta\phi$ is the difference in the azimuthal angle $\phi$ between the two particles and $\Delta\eta$ is the difference in pseudorapidity $\eta = -\ln(\tan(\theta/2))$. The polar angle $\theta$ is defined relative to the counterclockwise beam direction.

Of particular interest in studies of collective effects is the long-range (large $|\Delta\eta|$) structure of the two-particle correlation functions. In this region, the function is less susceptible to known sources of correlations such as resonance decays and fragmentation function of energetic jets. Measurements in high-energy AA collisions have shown significant modification of the long-range structure compared with minimum-bias pp collisions, over a very wide range of collision energies. The long-range correlations are commonly interpreted as a consequence of the hydrodynamical flow of the produced strongly interacting medium and usually characterized by the Fourier components of the azimuthal particle distributions. The extraction of the second and third Fourier components, usually referred to as elliptic and triangular flow, is of great interest because it is closely related to initial collision geometry and its fluctuation~\cite{Alver:2010gr}. Those measurements allow the extraction of the fundamental transport properties of the medium using hydrodynamic models.

Recently, measurements in pp collisions and pPb collisions have revealed the emergence of long-range, near-side ($\Delta\phi\sim 0$) correlations in the selection of collisions with very high number of final state particles. This ``ridge-like'' correlation has inspired a large variety of theoretical models~\cite{Bzdak:2013zma,Dusling:2015gta}. Moreover, it was found that the elliptic flow signal exists at the lowest nucleon-nucleon center-of-mass energy of 7.7 GeV in AA collisions at the Relativistic Heavy Ion Collider~\cite{Adamczyk:2012ku}. 

Studies of two-particle correlation in high multiplicity $e^+e^-$ will enable a direct comparison between different collision systems for the first time. Moreover, the total energy involved in the $e^+e^-$ collisions is known such that the initial condition is well-controlled. The studies of ridge signal in $e^+e^-$ collisions will bring significant impact to the field of relativistic heavy ion collisions, either change completely the interpretation of the ridge in pp, pA and AA collisions if a significant ridge signal is also observed, or serve as an important reference for the final state effect observed in high multiplicity hadron-hadron scatterings if no ridge signal was observed. The high-performance Belle detector is ideally suited for the proposed measurement.

Preliminary results from the Belle B-lab open data with the analysis method used in pA and AA collisions is presented in this proposal. Potential improvements of the analysis procedure could be implemented for the studies with full Belle hadronic data.
