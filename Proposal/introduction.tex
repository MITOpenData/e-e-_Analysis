\section{Introduction}

Two-particle correlations in high-energy collisions provide valuable information for characterizing Quantum Chromodynamics and have been studied previously for a broad range of collision energies in proton-proton (pp), proton-nucleus (pA), and nucleus-nucleus (AA) collisions. Such measurements can elucidate the underlying mechanism of particle production and reveal possible collective effects resulting from the high particle densities accessible in these collisions.

Studies of two-particle angular correlations are typically performed using two-dimensional $\Delta\eta-\Delta\phi$ correlation functions, where $\Delta\phi$ is the difference in the azimuthal angle $\phi$ between the two particles and $\Delta\eta$ is the difference in pseudorapidity $\eta = -\ln(\tan(\theta/2))$. The polar angle $\theta$ is defined relative to the counterclockwise beam direction.

Of particular interest in studies of collective effects is the long-range (large |$\delta\eta$|) structure of the two-particle correlation functions. In this region, the function is less susceptible to known sources of correlations such as resonance decays and fragmentation function of energetic jets. Measurements in high-energy AA collisions have shown significant modification of the long-range structure compared with minimum-bias pp collisions, over a very wide range of collision energies. This long-range correlations are commonly interpreted as a consequence of the hydrodynamical flow of the produced strongly interacting medium and usually characterized by the Fourier components of the azimuthal particle distributions. The extraction of the second and third Fourier components is of great interest because it is closely related to initial collision geometry and its fluctuation. Those measurements allows the extraction of the fundamental transport properties of the medium using hydrodynamic models.

Recently, measurements in pp collisions and pPb collisions have revealed the emergence of long-range, near-side ($\Delta\phi\sim 0$) correlations in the selection of collisions with very high number of final state particles. This ``ridge-like'' correlation has inspired a large variety of theoretical models. Moreover, it was found that the elliptic flow signal exists also in the lowest energy AA collisions at the Relativistic Heavy Ion Collider. 
