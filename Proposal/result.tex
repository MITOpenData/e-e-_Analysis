\section{Preliminary Results from Belle Open Data}

\subsection{Charged hadron multiplicity distributions}

Figure~\ref{fig:multPID} shows the multiplicity distribution of identified particles (pions, kaons and proton) obtained after 
applying the selection on the particle transverse momentum (0.1 $<p_{\rm T}<$ 4.0 GeV/$c$). 
The dominant contribution to the total multiplicity is coming as expected from pions.
In Fig.~\ref{fig:multHadron}, the charged hadron multiplicity distribution $N$ is shown: the largest raw particle multiplicity observed in these events is about 70 before corrections for tracking efficiency, fakes and multiple reconstruction rates. Detailed studies with primary vertex and track quality as well as possible pile-up event rejection are needed for the full Belle data analysis. 

\begin{figure}[!htb]
\begin{center}
\includegraphics[width=.32\textwidth]{figures/pion_mult.pdf}
\includegraphics[width=.32\textwidth]{figures/kaon_mult.pdf}
\includegraphics[width=.32\textwidth]{figures/proton_mult.pdf}
\caption{Uncorrected multiplicity distributions of pions (left), kaons (middle), protons (right) for  particles in the range  0.1 $<p_{\rm T}<$ 4.0 GeV/$c$ in $e^{+}e^{-}$ collisions. }
\label{fig:multPID} 
\end{center}
\end{figure}

\begin{figure}[!htb]
\begin{center}
\includegraphics[width=.45\textwidth]{figures/total_mult.pdf}
\caption{Multiplicity distribution of charged hadrons (protons, kaons and pions) for  particles in the range  0.1 $<p_{\rm T}<$ 4.0 GeV/$c$ in $e^{+}e^{-}$ collisions. }
\label{fig:multHadron} 
\end{center}
\end{figure}

\subsection{Two-particle correlation functions}


In Fig.~\ref{fig:ridgeBelle}, the two-particle correlation functions from low- (N$>$20) and high-multiplicity(N$>50$) events are presented. 
In low-multiplicity events, the dominant features of the correlation function are the jet peak near $(\Delta\eta,\Delta\phi)=(0,0)$ for pairs of particles originating from the same jet 
and the elongated structure at $\Delta\phi\sim\pi$ for pairs of particles from back-to-back jets. %To better illustrate the full correlation structure, the jet peak has been truncated in the high multiplicity event.
Moving from low-multiplicity to high-multiplicity selection, the same-side jet peak and back-to-back correlation structures are also observed. 
In addition, a hint of ``ridge"-like structure is visible at $\Delta\phi \sim$0 in the right panel of Fig.~\ref{fig:ridgeBelle}. 

To inspect the long-range and short-range structure further, one-dimensional distributions in $\Delta\phi$ are obtained by integrating over two $|\Delta\eta|$ intervals: 0$<|\Delta \eta|<$1 and 
2$<|\Delta \eta|<$3.  At small $\Delta\eta$, a near-side peak at $\Delta\phi=$0 and the contribution from the back-to-back jet at $\Delta\phi=\pi$ is observed in the left panel of Fig~\ref{fig:ProjectionMult50}. At large $\Delta\eta$, 
a near-side peak at $\Delta\phi=0$ is hinted (Fig~\ref{fig:ProjectionMult50}, right panel), similar to the structures observed in high multiplicity pp and pPb collisions at $\sqrt{s_{NN}}$ and in AA collisions over a wide range of energies. However, the significance of the signal is limited by the open data statistics. This preliminary observation motivates a detailed study with the high statistics data taken by the Belle collaboration. 

\begin{figure}[!htb]
\begin{center}
\includegraphics[width=.45\textwidth]{figures/canvasRidgeBelleMult20CutHigh0.pdf}
\includegraphics[width=.45\textwidth]{figures/canvasRidgeBelleMult50CutHigh0.pdf}
\caption{Two-particle correlation functions versus $\Delta\eta$ and $\Delta\phi$ in $e^{+}e^{-}$ collisions for events with particle multiplicity $>$ 20 (left) and  $>$ 50 (right).}
\label{fig:ridgeBelle} 
\end{center}
\end{figure}

\begin{figure}[!htb]
\begin{center}
\includegraphics[width=.45\textwidth]{figures/canvasProjection_isBelle1_mult50_eta01.pdf}
\includegraphics[width=.45\textwidth]{figures/canvasProjection_isBelle1_mult50_eta23.pdf}
\caption{Two-particle correlation functions as a function of  $\Delta\phi$ in $e^{+}e^{-}$ in the pseudorapidity ranges 0$<\Delta \eta<$1 (left) and 2$<\Delta \eta<$3 (right) for events with particle multiplicity $>$ 50.}
\label{fig:ProjectionMult50} 
\end{center}
\end{figure}

