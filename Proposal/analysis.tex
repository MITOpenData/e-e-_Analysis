\section{Two-Particle Correlation Function}

In this analysis with Belle open data, identified protons, pions and kaons with transverse momentum between 0.1 and 4.0 GeV/$c$ 
are selected for the correlation function analysis. High multiplicity events are sampled using the total number of selected proton, 
pions and kaons (hadron multiplicity $N$) in each event. The first step in extracting the correlation function was to divide the sample 
into bins in the hadron multiplicity. For each hadron multiplicity class, ``trigger" particles are defined as charged particles originating 
from primary vertex in the selected transverse momentum range (0.1 and 4.0 GeV/$c$). The number of trigger particles in the event is denoted
by $N_{trig}$. Particle pairs are then formed by associating every trigger particle with the remaining charged primary particles in the 
same $p_{\rm T}$ interval as the trigger particle. The per-trigger-particle associated yield is defined as:
\begin{eqnarray}
\label{eq:associatedyield}
\frac{1}{N_{\rm trig}}\frac{\rm d^2N^{pair}}{d\Delta\eta  \rm d\Delta\phi}= B(0,0) \times \frac{S(\Delta\eta, \Delta\phi)}{B(\Delta\eta, \Delta\phi)}
\end{eqnarray}
where $\Delta\eta$ and $\Delta\phi$ are the differences in $\eta$ and $\phi$ of the pair. The signal distribution, $S(\Delta\eta, \Delta\phi)$, 
is the per-trigger-particle yield of particle pairs in the same event: 
\begin{eqnarray}
\label{eq:S}
S(\Delta\eta,\Delta\phi) = \frac{1}{N_{trig}}\frac{\rm d^2 N^{\rm same}}{\rm d\Delta\eta \rm d\Delta\phi}
\end{eqnarray}
The mixed-event background distribution, used to account for random combinatorial background, is defined as 
\begin{eqnarray}
\label{eq:B}
B(\Delta\eta,\Delta\phi) = \frac{1}{N_{trig}}\frac{\rm d^2 N^{\rm mix}}{\rm d\Delta\eta \rm d\Delta\phi}
\end{eqnarray}
and is constructing by pairing the trigger particles from two random events in the same hadron multiplicity interval.
The symbol $N^{mix}$ denotes the number of pairs taken from the mixed event, while $B(0,0)$ represents the 
mixed-event associated yield for both particles of the pair going in the same direction and thus having full pair acceptance. Therefore, 
the ratio $B(0,0)/B(\Delta\eta,\Delta\phi)$ represents the pair-acceptance correction factor used to derive the corrected per-trigger-particle
associated yield distribution.  The signal and background distributions are first calculated for each event and then averaged over all the events 
within the track multiplicity class.
