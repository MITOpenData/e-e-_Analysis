\section{The ALEPH Detector}

The central part of the ALEPH detector is dedicated to the reconstruction of charged particles, which are measured by a two-layer silicon strip vertex detector, a cylindrical drift chamber and a large time projection chamber. The three detectors are immersed in a 1.5 T axial magnetic field generated by a superconducting solenoid. Electrons and photons are identified in the electromagnetic calorimeter (ECAL), which is a sampling calorimeter sandwidch of lead plates and proportional wire chambers segmented in $0.9^\circ\times 0.9^\circ$ projective towers and read out in three sections in depth. The iron return yoke forms the hadron calorimeter (HCAL), which are used to measure the energy of charged and neutral hadrons. Muons are distinguished from hadrons by HCAL and muon chambers, which consists of two double-layers of streamer tubes outside HCAL. Detailed description of the ALEPH detector and its performance could be found in ~\ref{}.
